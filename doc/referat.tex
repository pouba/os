\documentclass[a4paper]{article}
\usepackage[czech]{babel}

\usepackage[utf8]{inputenc}

\usepackage{graphicx}
\usepackage{float}
%\usepackage{pdfpages}
%\input kvmacros

\begin{document}

\begin{titlepage}

\begin{center}

% tituln strana
\author{Marek Šimůnek}

\includegraphics[width=0.5\textwidth]{./logo.jpg} \\[1cm]
 
 \huge Semestrální práce z  KIV/OS \\ [1.5cm]
\huge \textbf{ Simulace operačního systému}\\[0.9cm]
{\Large \textsc{Marek Šimůnek} } {\large \textsc{A15B0082P}  } 

{\Large \textsc{Jindřich Pouba }}{\large \textsc{A15B0072P}  } 

{\Large \textsc{Matěj Lochman }}{\large \textsc{A15B0068P}  } 
\\[0.9cm]




\large \textsc  {\today}



\end{center}


\end{titlepage}

\section{Zadání}
\begin{itemize}
\item Vytvořte virtuální stroj, který bude simulovat OS
\item Součástí bude shell s gramatikou cmd
\item Vytvoříte ekvivalenty standardních příkazů a programů
\begin{itemize}
\item echo, cd, dir, md, rd, type, wc, sort
\item Dále vytvoříte programy rand a freq
\item rand bude vypisovat náhodně vygenerovaná čísla v plovoucí čárce na stdout, dokud mu nepřijde znak Ctrl+Z //EOF
\item freq bude číst z stdin a sestaví frekvenční tabulku bytů, kterou pak vypíše pro všechny byty s frekvencí větší než 0 ve formátu: 

“0x\%hhx : \%d”
\end{itemize}


\item Implementujte roury a přesměrování
\item Nebudete přistupovat na souborový systém, ale uděláte si prostředky simulátoru vlastní RAM-disk s názvem C
\end{itemize}

%\begin{figure*}[] 
%\centering
% \makebox[\textwidth]{\includegraphics[width=.9\paperwidth]{./pidalkaguide.pdf}}
%\label{fig:Fig5}
%\end{figure*}

%\includepdf{./pidalkaguide.pdf}

%\begin{itemize}
%\item Seznamte se s konfigurací reálného modelu robotu s podtlakovým přisáváním.
%\item Navrhněte algoritmy pro jednotlivé elementární kroky Píďalky (minimálně dopředu, dozadu, vlevo, vpravo).
%\item Navrhněte grafické rozhraní, pomocí kterého bude možné systém ovládat z počítače.
%\item (volitelné) Implementujte řídicí algoritmus, pomocí kterého vykoná Píďalka předem definovanou sekvenci kroků.
%\end{itemize}

\section{Implementace}

\section{Uživatelská příručka}
Pro spuštění programu stačí otevřít spustitelný soubor \emph{os.exe}. V terminálovém okně se zobrazí aktuální adresář, ve kterém se nacházíte. V tuto chvíli můžete zadávat příkazy.
 Implementovány jsou příkazy ze zadání a navíc jsou ještě přidány příkazy: 
\begin{itemize}
\item  \verb+scan+, který vypisuje ascii hodnotu ze vstupu
\item  \verb+pipe+, který vypisuje vstup na výstup
\end{itemize}

Každý příkaz může být přesměrován na výstup zápisem  \verb+příkaz+ \textgreater  \verb+ soubor+ a na vstup  \verb+příkaz+ \textless  \verb+ soubor+. Lze i přesměrovat na konec souboru konstrukcí \textgreater\textgreater. Příkazy je dále možné spojovat pomocí rour pro nasměrování výstupu jednoho procesu na vstup následujícího procesu  \verb+příkaz1 | příkaz2+.

Ukončení čtení se provádí CTRL+Z. 
V spuštěné příkazové řádce lze spustit další příkazovou řádku. Každá příkazová řádka se ukončuje příkazem \verb+exit+. 



\section{Závěr}





\end{document}
